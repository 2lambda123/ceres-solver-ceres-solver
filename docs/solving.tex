%!TEX root = ceres-solver.tex
\chapter{Solving}
Effective use of Ceres requires some familiarity with the basic components of a nonlinear least squares solver, so before we describe how to configure the solver, we will begin by taking a brief look at how some of the core optimization algorithms in Ceres work and the various linear solvers and preconditioners that power it.

\section{Trust Region Methods}
\label{sec:trust-region}
Let $x \in \mathbb{R}^{n}$ be an $n$-dimensional vector of variables, and
$ F(x) = \left[f_1(x),   \hdots,  f_{m}(x) \right]^{\top}$ be a $m$-dimensional function of $x$.  We are interested in solving the following optimization problem~\footnote{At the level of the non-linear solver, the block and residual structure is not relevant, therefore our discussion here is in terms of an optimization problem defined over a state vector of size $n$.},
\begin{equation}
        \arg \min_x \frac{1}{2}\|F(x)\|^2\ .
        \label{eq:nonlinsq}
\end{equation}
Here, the Jacobian $J(x)$ of $F(x)$ is an $m\times n$ matrix, where $J_{ij}(x) = \partial_j f_i(x)$  and the gradient vector $g(x) = \nabla  \frac{1}{2}\|F(x)\|^2 = J(x)^\top F(x)$. Since the efficient global optimization of~\eqref{eq:nonlinsq} for general $F(x)$ is an intractable problem, we will have to settle for finding a local minimum.

The general strategy when solving non-linear optimization problems is to solve a sequence of approximations to the original problem~\cite{nocedal2000numerical}. At each iteration, the approximation is solved to determine a correction $\Delta x$ to the vector $x$. For non-linear least squares, an approximation can be constructed by using the linearization $F(x+\Delta x) \approx F(x) + J(x)\Delta x$, which leads to the following linear least squares  problem:
\begin{equation}
         \min_{\Delta x} \frac{1}{2}\|J(x)\Delta x + F(x)\|^2
        \label{eq:linearapprox}
\end{equation}
Unfortunately, na\"ively solving a sequence of these problems and
updating $x \leftarrow x+ \Delta x$ leads to an algorithm that may not
converge.  To get a convergent algorithm, we need to control the size
of the step $\Delta x$. And this is where the idea of a trust-region
comes in. Algorithm~\ref{alg:trust-region} describes the basic  trust-region loop for non-linear least squares problems.

\begin{algorithm}
\caption{The basic trust-region algorithm.\label{alg:trust-region}}
\begin{algorithmic}
\REQUIRE Initial point $x$ and a trust region radius $\mu$.
\LOOP
\STATE{Solve $\arg \min_{\Delta x} \frac{1}{2}\|J(x)\Delta x + F(x)\|^2$ s.t. $\|D(x)\Delta x\|^2 \le \mu$}
\STATE{$\rho = \frac{\displaystyle \|F(x + \Delta x)\|^2 - \|F(x)\|^2}{\displaystyle \|J(x)\Delta x + F(x)\|^2 - \|F(x)\|^2}$}
\IF {$\rho > \epsilon$}
\STATE{$x = x + \Delta x$}
\ENDIF
\IF {$\rho > \eta_1$}
\STATE{$\rho = 2 * \rho$}
\ELSE
\IF {$\rho < \eta_2$}
\STATE {$\rho = 0.5 * \rho$}
\ENDIF
\ENDIF
\ENDLOOP
\end{algorithmic}
\end{algorithm}

Here, $\mu$ is the trust region radius, $D(x)$ is some matrix used to define a metric on the domain of $F(x)$ and $\rho$ measures the quality of the step $\Delta x$, i.e., how well did the linear model predict the decrease in the value of the non-linear objective. The idea is to increase or decrease the radius of the trust region depending on how well the linearization predicts the behavior of the non-linear objective, which in turn is reflected in the value of $\rho$.

The key computational step in a trust-region algorithm is the solution of the constrained optimization problem
\begin{align}
        \arg\min_{\Delta x}& \frac{1}{2}\|J(x)\Delta x + F(x)\|^2 \\
        \text{such that}&\quad  \|D(x)\Delta x\|^2 \le \mu
\label{eq:trp}
\end{align}

There are a number of different ways of solving this problem, each giving rise to a different concrete trust-region algorithm. Currently Ceres, implements two trust-region algorithms - Levenberg-Marquardt and  Dogleg.

\subsection{Levenberg-Marquardt}
The Levenberg-Marquardt algorithm~\cite{levenberg1944method, marquardt1963algorithm} is the most popular algorithm for solving non-linear least squares problems.  It was also the first trust region algorithm to be developed~\cite{levenberg1944method,marquardt1963algorithm}. Ceres implements an exact step~\cite{madsen2004methods} and an inexact step variant of the Levenberg-Marquardt algorithm~\cite{wright1985inexact,nash1990assessing}.

It can be shown, that the solution to~\eqref{eq:trp} can be obtained by solving an unconstrained optimization of the form
\begin{align}
        \arg\min_{\Delta x}& \frac{1}{2}\|J(x)\Delta x + F(x)\|^2 +\lambda  \|D(x)\Delta x\|^2
\end{align}
Where, $\lambda$ is a Lagrange multiplier that is inverse related to $\mu$. In Ceres, we solve for
\begin{align}
        \arg\min_{\Delta x}& \frac{1}{2}\|J(x)\Delta x + F(x)\|^2 + \frac{1}{\mu} \|D(x)\Delta x\|^2
\label{eq:lsqr}
\end{align}
The matrix $D(x)$ is a non-negative diagonal matrix, typically the square root of the diagonal of the matrix $J(x)^\top J(x)$.

Before going further, let us make some notational simplifications. We will assume that the matrix $\sqrt{\mu} D$ has been concatenated at the bottom of the matrix $J$ and similarly a vector of zeros has been added to the bottom of the vector $f$ and the rest of our discussion will be in terms of $J$ and $f$, \ie the linear least squares problem.
\begin{align}
 \min_{\Delta x} \frac{1}{2} \|J(x)\Delta x + f(x)\|^2 .
 \label{eq:simple}
\end{align}
For all but the smallest problems the solution of~\eqref{eq:simple} in each iteration of the Levenberg-Marquardt algorithm is the dominant computational cost in Ceres. Ceres provides a number of different options for solving~\eqref{eq:simple}. There are two major classes of methods - factorization and iterative.

The factorization methods are based on computing an exact solution of~\eqref{eq:lsqr} using a Cholesky or a QR factorization and lead to an exact step Levenberg-Marquardt algorithm. But it is not clear if an exact solution of~\eqref{eq:lsqr} is necessary at each step of the LM algorithm to solve~\eqref{eq:nonlinsq}. In fact, we have already seen evidence that this may not be the case, as~\eqref{eq:lsqr} is itself a regularized version of~\eqref{eq:linearapprox}. Indeed, it is possible to construct non-linear optimization algorithms in which the linearized problem is solved approximately. These algorithms are known as inexact Newton or truncated Newton methods~\cite{nocedal2000numerical}.

An inexact Newton method requires two ingredients. First, a cheap method for approximately solving systems of linear equations. Typically an iterative linear solver like the Conjugate Gradients method is used for this purpose~\cite{nocedal2000numerical}. Second, a termination rule for the iterative solver. A typical termination rule is of the form
\begin{equation}
        \|H(x) \Delta x + g(x)\| \leq \eta_k \|g(x)\|. \label{eq:inexact}
\end{equation}
Here, $k$ indicates the Levenberg-Marquardt iteration number and $0 < \eta_k <1$ is known as the forcing sequence.  Wright \& Holt \cite{wright1985inexact} prove that a truncated Levenberg-Marquardt algorithm that uses an inexact Newton step based on~\eqref{eq:inexact} converges for any sequence $\eta_k \leq \eta_0 < 1$ and the rate of convergence depends on the choice of the forcing sequence $\eta_k$.

Ceres supports both exact and inexact step solution strategies. When the user chooses a factorization based linear solver, the exact step Levenberg-Marquardt algorithm is used. When the user chooses an iterative linear solver, the inexact step Levenberg-Marquardt algorithm is used.

\subsection{Dogleg}
\label{sec:dogleg}
Another strategy for solving the trust region problem~\eqref{eq:trp} was introduced by M. J. D. Powell. The key idea there is to compute two vectors
\begin{align}
        \Delta x^{\text{Gauss-Newton}} &= \arg \min_{\Delta x}\frac{1}{2} \|J(x)\Delta x + f(x)\|^2.\\
        \Delta x^{\text{Cauchy}} &= -\frac{\|g(x)\|^2}{\|J(x)g(x)\|^2}g(x).
\end{align}
Note that the vector $\Delta x^{\text{Gauss-Newton}}$ is the solution
to~\eqref{eq:linearapprox} and $\Delta x^{\text{Cauchy}}$ is the
vector that minimizes the linear approximation if we restrict
ourselves to moving along the direction of the gradient. Dogleg methods finds a vector $\Delta x$ defined by $\Delta
x^{\text{Gauss-Newton}}$ and $\Delta x^{\text{Cauchy}}$ that solves
the trust region problem. Ceres supports two
variants.

\texttt{TRADITIONAL\_DOGLEG} as described by Powell,
constructs two line segments using the Gauss-Newton and Cauchy vectors
and finds the point farthest along this line shaped like a dogleg
(hence the name) that is contained in the
trust-region. For more details on the exact reasoning and computations, please see Madsen et al~\cite{madsen2004methods}.

 \texttt{SUBSPACE\_DOGLEG} is a more sophisticated method
that considers the entire two dimensional subspace spanned by these
two vectors and finds the point that minimizes the trust region
problem in this subspace\cite{byrd1988approximate}.

The key advantage of the Dogleg over Levenberg Marquardt is that if the step computation for a particular choice of $\mu$ does not result in sufficient decrease in the value of the objective function, Levenberg-Marquardt solves the linear approximation from scratch with a smaller value of $\mu$. Dogleg on the other hand, only needs to compute the interpolation between the Gauss-Newton and the Cauchy vectors, as neither of them depend on the value of $\mu$.

The Dogleg method can only be used with the exact factorization based linear solvers.

\subsection{Inner Iterations}
\label{sec:inner}
Some non-linear least squares problems have additional structure in
the way the parameter blocks interact that it is beneficial to modify
the way the trust region step is computed. e.g., consider the
following regression problem

\begin{equation}
  y = a_1 e^{b_1 x} + a_2 e^{b_3 x^2 + c_1}
\end{equation}

Given a set of pairs $\{(x_i, y_i)\}$, the user wishes to estimate
$a_1, a_2, b_1, b_2$, and $c_1$.

Notice that the expression on the left is linear in $a_1$ and $a_2$,
and given any value for $b_1$, $b_2$ and $c_1$, it is possible to use
linear regression to estimate the optimal values of $a_1$ and
$a_2$. It's possible to analytically eliminate the variables
$a_1$ and $a_2$ from the problem entirely. Problems like these are
known as separable least squares problem and the most famous algorithm
for solving them is the Variable Projection algorithm invented by
Golub \& Pereyra~\cite{golub-pereyra-73}.

Similar structure can be found in the matrix factorization with
missing data problem. There the corresponding algorithm is
known as Wiberg's algorithm~\cite{wiberg}.

Ruhe \& Wedin  present an analysis of
various algorithms for solving separable non-linear least
squares problems and refer to {\em Variable Projection} as
Algorithm I in their paper~\cite{ruhe-wedin}.

Implementing Variable Projection is tedious and expensive. Ruhe \&
Wedin present a simpler algorithm with comparable convergence
properties, which they call Algorithm II.  Algorithm II performs an
additional optimization step to estimate $a_1$ and $a_2$ exactly after
computing a successful Newton step.


This idea can be generalized to cases where the residual is not
linear in $a_1$ and $a_2$, i.e.,

\begin{equation}
  y = f_1(a_1, e^{b_1 x}) + f_2(a_2, e^{b_3 x^2 + c_1})
\end{equation}

In this case, we solve for the trust region step for the full problem,
and then use it as the starting point to further optimize just $a_1$
and $a_2$. For the linear case, this amounts to doing a single linear
least squares solve. For non-linear problems, any method for solving
the $a_1$ and $a_2$ optimization problems will do. The only constraint
on $a_1$ and $a_2$ (if they are two different parameter block) is that
they do not co-occur in a residual block.

This idea can be further generalized, by not just optimizing $(a_1,
a_2)$, but decomposing the graph corresponding to the Hessian matrix's
sparsity structure into a collection of non-overlapping independent sets
and optimizing each of them.

Setting \texttt{Solver::Options::use\_inner\_iterations} to true
enables
the use of this non-linear generalization of Ruhe \& Wedin's Algorithm
II.  This version of Ceres has a higher iteration complexity, but also
displays better convergence behavior per iteration.

Setting \texttt{Solver::Options::num\_threads} to the maximum number
possible is highly recommended.

\subsection{Non-monotonic Steps}
\label{sec:non-monotonic}
Note that the basic trust-region algorithm described in
Algorithm~\ref{alg:trust-region} is a descent algorithm  in that they
only accepts a point if it strictly reduces the value of the objective
function.

Relaxing this requirement allows the algorithm to be more
efficient in the long term at the cost of some local increase
in the value of the objective function.

This is because allowing for non-decreasing objective function
values in a princpled manner allows the algorithm to ``jump over
boulders'' as the method is not restricted to move into narrow
valleys while preserving its convergence properties.

Setting \texttt{Solver::Options::use\_nonmonotonic\_steps} to \texttt{true}
enables the non-monotonic trust region algorithm as described by
Conn,  Gould \& Toint in~\cite{conn2000trust}.

Even though the value of the objective function may be larger
than the minimum value encountered over the course of the
optimization, the final parameters returned to the user are the
ones corresponding to the minimum cost over all iterations.

The option to take non-monotonic is available for all trust region
strategies.

\section{\texttt{LinearSolver}}
Recall that in both of the trust-region methods described above, the key computational cost is the solution of a linear least squares problem of the form
\begin{align}
 \min_{\Delta x} \frac{1}{2} \|J(x)\Delta x + f(x)\|^2 .
 \label{eq:simple2}
\end{align}


Let $H(x)= J(x)^\top J(x)$ and $g(x) = -J(x)^\top  f(x)$. For notational convenience let us also drop the dependence on $x$. Then it is easy to see that solving~\eqref{eq:simple2} is equivalent to solving the {\em normal equations}
\begin{align}
H \Delta x  &= g \label{eq:normal}
\end{align}

Ceres provides a number of different options for solving~\eqref{eq:normal}.

\subsection{\texttt{DENSE\_QR}}
For small problems (a couple of hundred parameters and a few thousand residuals) with relatively dense Jacobians, \texttt{DENSE\_QR} is the method of choice~\cite{bjorck1996numerical}. Let $J = QR$ be the QR-decomposition of $J$, where $Q$ is an orthonormal matrix and $R$ is an upper triangular matrix~\cite{trefethen1997numerical}. Then it can be shown that the solution to~\eqref{eq:normal} is given by
\begin{align}
    \Delta x^* = -R^{-1}Q^\top f
\end{align}
Ceres uses \texttt{Eigen}'s dense QR factorization routines.

\subsection{\texttt{DENSE\_NORMAL\_CHOLESKY} \& \texttt{SPARSE\_NORMAL\_CHOLESKY}}
Large non-linear least square problems are usually sparse. In such cases, using a dense QR factorization is inefficient. Let $H = R^\top R$ be the Cholesky factorization of the normal equations, where $R$ is an upper triangular matrix, then the  solution to ~\eqref{eq:normal} is given by
\begin{equation}
    \Delta x^* = R^{-1} R^{-\top} g.
\end{equation}
The observant reader will note that the $R$ in the Cholesky
factorization of $H$ is the same upper triangular matrix $R$ in the QR
factorization of $J$. Since $Q$ is an orthonormal matrix, $J=QR$
implies that $J^\top J = R^\top Q^\top Q R = R^\top R$. There are two variants of Cholesky factorization -- sparse and
dense.

\texttt{DENSE\_NORMAL\_CHOLESKY}  as the name implies performs a dense
Cholesky factorization of the normal equations. Ceres uses
\texttt{Eigen}'s dense LDLT factorization routines.

\texttt{SPARSE\_NORMAL\_CHOLESKY}, as the name implies performs a
sparse Cholesky factorization of the normal equations. This leads to
substantial savings in time and memory for large sparse
problems. Ceres uses the sparse Cholesky factorization routines in  Professor Tim Davis'  \texttt{SuiteSparse} or
\texttt{CXSparse} packages~\cite{chen2006acs}.

\subsection{\texttt{DENSE\_SCHUR} \& \texttt{SPARSE\_SCHUR}}
While it is possible to use \texttt{SPARSE\_NORMAL\_CHOLESKY} to solve bundle adjustment problems, bundle adjustment problem have a special structure, and a more efficient scheme for solving~\eqref{eq:normal} can be constructed.

Suppose that the SfM problem consists of $p$ cameras and $q$ points and the variable vector $x$ has the  block structure $x = [y_{1},\hdots,y_{p},z_{1},\hdots,z_{q}]$. Where, $y$ and $z$ correspond to camera and point parameters, respectively.  Further, let the camera blocks be of size $c$ and the point blocks be of size $s$ (for most problems $c$ =  $6$--$9$ and $s = 3$). Ceres does not impose any constancy requirement on these block sizes, but choosing them to be constant simplifies the exposition.

A key characteristic of the bundle adjustment problem is that there is no term $f_{i}$ that includes two or more point blocks.  This in turn implies that the matrix $H$ is of the form
\begin{equation}
        H =  \left[
                \begin{matrix} B & E\\ E^\top & C
                \end{matrix}
                \right]\ ,
\label{eq:hblock}
\end{equation}
where, $B \in \reals^{pc\times pc}$ is a block sparse matrix with $p$ blocks of size $c\times c$ and  $C \in \reals^{qs\times qs}$ is a block diagonal matrix with $q$ blocks of size $s\times s$. $E \in \reals^{pc\times qs}$ is a general block sparse matrix, with a block of size $c\times s$ for each observation. Let us now block partition $\Delta x = [\Delta y,\Delta z]$ and $g=[v,w]$ to restate~\eqref{eq:normal} as the block structured linear system
\begin{equation}
        \left[
                \begin{matrix} B & E\\ E^\top & C
                \end{matrix}
                \right]\left[
                        \begin{matrix} \Delta y \\ \Delta z
                        \end{matrix}
                        \right]
                        =
                        \left[
                                \begin{matrix} v\\ w
                                \end{matrix}
                                \right]\ ,
\label{eq:linear2}
\end{equation}
and apply Gaussian elimination to it. As we noted above, $C$ is a block diagonal matrix, with small diagonal blocks of size $s\times s$.
Thus, calculating the inverse of $C$ by inverting each of these blocks is  cheap. This allows us to  eliminate $\Delta z$ by observing that $\Delta z = C^{-1}(w - E^\top \Delta y)$, giving us
\begin{equation}
        \left[B - EC^{-1}E^\top\right] \Delta y = v - EC^{-1}w\ .  \label{eq:schur}
\end{equation}
The matrix
\begin{equation}
S = B - EC^{-1}E^\top\ ,
\end{equation}
is the Schur complement of $C$ in $H$. It is also known as the {\em reduced camera matrix}, because the only variables participating in~\eqref{eq:schur} are the ones corresponding to the cameras. $S \in \reals^{pc\times pc}$ is a block structured symmetric positive definite matrix, with blocks of size $c\times c$. The block $S_{ij}$ corresponding to the pair of images $i$ and $j$ is non-zero if and only if the two images observe at least one common point.

Now, \eqref{eq:linear2}~can  be solved by first forming $S$, solving for $\Delta y$, and then back-substituting $\Delta y$ to obtain the value of $\Delta z$.
Thus, the solution of what was an $n\times n$, $n=pc+qs$ linear system is reduced to the inversion of the block diagonal matrix $C$, a few matrix-matrix and matrix-vector multiplies, and the solution of block sparse $pc\times pc$ linear system~\eqref{eq:schur}.  For almost all  problems, the number of cameras is much smaller than the number of points, $p \ll q$, thus solving~\eqref{eq:schur} is significantly cheaper than solving~\eqref{eq:linear2}. This is the {\em Schur complement trick}~\cite{brown-58}.

This still leaves open the question of solving~\eqref{eq:schur}. The
method of choice for solving symmetric positive definite systems
exactly is via the Cholesky
factorization~\cite{trefethen1997numerical} and depending upon the
structure of the matrix, there are, in general, two options. The first
is direct factorization, where we store and factor $S$ as a dense
matrix~\cite{trefethen1997numerical}. This method has $O(p^2)$ space complexity and $O(p^3)$ time
complexity and is only practical for problems with up to a few hundred
cameras. Ceres implements this strategy as the \texttt{DENSE\_SCHUR} solver.


 But, $S$ is typically a fairly sparse matrix, as most images
only see a small fraction of the scene. This leads us to the second
option: sparse direct methods. These methods store $S$ as a sparse
matrix, use row and column re-ordering algorithms to maximize the
sparsity of the Cholesky decomposition, and focus their compute effort
on the non-zero part of the factorization~\cite{chen2006acs}.
Sparse direct methods, depending on the exact sparsity structure of the Schur complement,
allow bundle adjustment algorithms to significantly scale up over those based on dense
factorization. Ceres implements this strategy as the \texttt{SPARSE\_SCHUR} solver.

\subsection{\texttt{CGNR}}
For general sparse problems, if the problem is too large for \texttt{CHOLMOD} or a sparse linear algebra library is not linked into Ceres, another option is the \texttt{CGNR} solver. This solver uses the Conjugate Gradients solver on the {\em normal equations}, but without forming the normal equations explicitly. It exploits the relation
\begin{align}
    H x = J^\top J x = J^\top(J x)
\end{align}
When the user chooses \texttt{ITERATIVE\_SCHUR} as the linear solver, Ceres automatically switches from the exact step algorithm to an inexact step algorithm.

\subsection{\texttt{ITERATIVE\_SCHUR}}
Another option for bundle adjustment problems is to apply PCG to the reduced camera matrix $S$ instead of $H$. One reason to do this is that $S$ is a much smaller matrix than $H$, but more importantly, it can be shown that $\kappa(S)\leq \kappa(H)$.  Ceres implements PCG on $S$ as the \texttt{ITERATIVE\_SCHUR} solver. When the user chooses \texttt{ITERATIVE\_SCHUR} as the linear solver, Ceres automatically switches from the exact step algorithm to an inexact step algorithm.

The cost of forming and storing the Schur complement $S$ can be prohibitive for large problems. Indeed, for an inexact Newton solver that computes $S$ and runs PCG on it, almost all of its time is spent in constructing $S$; the time spent inside the PCG algorithm is negligible in comparison. Because  PCG only needs access to $S$ via its product with a vector, one way to evaluate $Sx$ is to observe that
\begin{align}
  x_1 &= E^\top x \notag \\
  x_2 &= C^{-1} x_1 \notag\\
  x_3 &= Ex_2 \notag\\
  x_4 &= Bx \notag\\
  Sx &= x_4 - x_3\ .\label{eq:schurtrick1}
\end{align}
Thus, we can run PCG on $S$ with the same computational effort per iteration as PCG on $H$, while reaping the benefits of a more powerful preconditioner. In fact, we do not even need to compute $H$, \eqref{eq:schurtrick1} can be implemented using just the columns of $J$.

Equation~\eqref{eq:schurtrick1} is closely related to {\em Domain Decomposition methods} for solving large linear systems that arise in structural engineering and partial differential equations. In the language of Domain Decomposition, each point in a bundle adjustment problem is a domain, and the cameras form the interface between these domains. The iterative solution of the Schur complement then falls within the sub-category of techniques known as Iterative Sub-structuring~\cite{saad2003iterative,mathew2008domain}.

\section{Preconditioner}
The convergence rate of Conjugate Gradients  for solving~\eqref{eq:normal} depends on the distribution of eigenvalues of $H$~\cite{saad2003iterative}. A useful upper bound is $\sqrt{\kappa(H)}$, where, $\kappa(H)$f is the condition number of the matrix $H$. For most bundle adjustment problems, $\kappa(H)$ is high and a direct application of Conjugate Gradients to~\eqref{eq:normal} results in extremely poor performance.

The solution to this problem is to replace~\eqref{eq:normal} with a {\em preconditioned} system.  Given a linear system, $Ax =b$ and a preconditioner $M$ the preconditioned system is given by $M^{-1}Ax = M^{-1}b$. The resulting algorithm is known as Preconditioned Conjugate Gradients algorithm (PCG) and its  worst case complexity now depends on the condition number of the {\em preconditioned} matrix $\kappa(M^{-1}A)$.

The computational cost of using a preconditioner $M$ is the cost of computing $M$ and evaluating the product $M^{-1}y$ for arbitrary vectors $y$. Thus, there are two competing factors to consider: How much of $H$'s structure is captured by $M$ so that the condition number $\kappa(HM^{-1})$ is low, and the computational cost of constructing and using $M$.  The ideal preconditioner would be one for which $\kappa(M^{-1}A) =1$. $M=A$ achieves this, but it is not a practical choice, as applying this preconditioner would require solving a linear system equivalent to the unpreconditioned problem.  It is usually the case that the more information $M$ has about $H$, the more expensive it is use. For example, Incomplete Cholesky factorization based preconditioners  have much better convergence behavior than the Jacobi preconditioner, but are also much more expensive.


The simplest of all preconditioners is the diagonal or Jacobi preconditioner, \ie,  $M=\operatorname{diag}(A)$, which for block structured matrices like $H$ can be generalized to the block Jacobi preconditioner.

For \texttt{ITERATIVE\_SCHUR} there are two obvious choices for block diagonal preconditioners for $S$. The block diagonal of the matrix $B$~\cite{mandel1990block} and the block diagonal $S$, \ie the block Jacobi preconditioner for $S$. Ceres's implements both of these preconditioners and refers to them as  \texttt{JACOBI} and \texttt{SCHUR\_JACOBI} respectively.

For bundle adjustment problems arising in reconstruction from community photo collections, more effective preconditioners can be constructed by analyzing and exploiting the camera-point visibility structure of the scene~\cite{kushal2012}. Ceres implements the two visibility based preconditioners described by Kushal \& Agarwal as \texttt{CLUSTER\_JACOBI} and \texttt{CLUSTER\_TRIDIAGONAL}. These are fairly new preconditioners and Ceres' implementation of them is in its early stages and is not as mature as the other preconditioners described above.

\section{Ordering}
\label{sec:ordering}
The order in which variables are eliminated in a linear solver can
have a significant of impact on the efficiency and accuracy of the
method. For example when doing sparse Cholesky factorization, there are
matrices for which a good ordering will give a Cholesky factor with
O(n) storage, where as a bad ordering will result in an completely
dense factor.

Ceres allows the user to provide varying amounts of hints to the
solver about the variable elimination ordering to use. This can range
from no hints, where the solver is free to decide the best ordering
based on the user's choices like the linear solver being used, to an
exact order in which the variables should be eliminated, and a variety
of possibilities in between.

Instances of the \texttt{Ordering} class are used to communicate this
information to Ceres.

Formally an ordering is an ordered partitioning of the parameter
blocks. Each parameter block belongs to exactly one group, and
each group has a unique integer associated with it, that determines
its order in the set of groups. We call these groups {\em elimination
  groups}.

Given such an ordering, Ceres ensures that the parameter blocks in the
lowest numbered elimination group are eliminated first, and then the
parameter blocks in the next lowest numbered elimination group and so
on. Within each elimination group, Ceres is free to order the
parameter blocks as it chooses. e.g. Consider the linear system

\begin{align}
x + y &= 3\\
   2x + 3y &= 7
\end{align}

There are two ways in which it can be solved. First eliminating $x$
from the two equations, solving for y and then back substituting
for $x$, or first eliminating $y$, solving for $x$ and back substituting
for $y$. The user can construct three orderings here.

\begin{enumerate}
\item   $\{0: x\}, \{1: y\}$: Eliminate $x$ first.
\item  $\{0: y\}, \{1: x\}$: Eliminate $y$ first.
\item   $\{0: x, y\}$: Solver gets to decide the elimination order.
\end{enumerate}

Thus, to have Ceres determine the ordering automatically using
heuristics, put all the variables in the same elimination group. The
identity of the group does not matter. This is the same as not
specifying an ordering at all. To control the ordering for every
variable, create an elimination group per variable, ordering them in
the desired order.

If the user is using one of the Schur solvers (\texttt{DENSE\_SCHUR},
\texttt{SPARSE\_SCHUR},\ \texttt{ITERATIVE\_SCHUR}) and chooses to
specify an ordering, it must have one important property. The lowest
numbered elimination group must form an independent set in the graph
corresponding to the Hessian, or in other words, no two parameter
blocks in in the first elimination group should co-occur in the same
residual block. For the best performance, this elimination group
should be as large as possible. For standard bundle adjustment
problems, this corresponds to the first elimination group containing
all the 3d points, and the second containing the all the cameras
parameter blocks.

If the user leaves the choice to Ceres, then the solver uses an
approximate maximum independent set algorithm to identify the first
elimination group~\cite{li2007miqr} .
\section{\texttt{Solver::Options}}

\texttt{Solver::Options} controls the overall behavior of the
solver. We list the various settings and their default values below.

\begin{enumerate}

\item{\texttt{trust\_region\_strategy\_type }}
  (\texttt{LEVENBERG\_MARQUARDT}) The  trust region step computation
  algorithm used by Ceres. Currently \texttt{LEVENBERG\_MARQUARDT }
  and \texttt{DOGLEG} are the two valid choices.

\item{\texttt{dogleg\_type}} (\texttt{TRADITIONAL\_DOGLEG})  Ceres
  supports two different dogleg strategies.
  \texttt{TRADITIONAL\_DOGLEG} method by Powell and the
  \texttt{SUBSPACE\_DOGLEG} method described by Byrd et al.
~\cite{byrd1988approximate}. See Section~\ref{sec:dogleg} for more details.

\item{\texttt{use\_nonmonotoic\_steps}} (\texttt{false})
Relax the requirement that the trust-region algorithm take strictly
decreasing steps. See Section~\ref{sec:non-monotonic} for more details.

\item{\texttt{max\_consecutive\_nonmonotonic\_steps}} (5)
The window size used by the step selection algorithm to accept
non-monotonic steps.

\item{\texttt{max\_num\_iterations }}(\texttt{50}) Maximum number of
  iterations for Levenberg-Marquardt.

\item{\texttt{max\_solver\_time\_in\_seconds }} ($10^9$) Maximum
  amount of time for which the solver should run.

\item{\texttt{num\_threads }}(\texttt{1}) Number of threads used by
  Ceres to evaluate the Jacobian.

\item{\texttt{initial\_trust\_region\_radius } ($10^4$)} The size of
  the initial trust region. When the \texttt{LEVENBERG\_MARQUARDT}
  strategy is used, the reciprocal of this number is the initial
  regularization parameter.

\item{\texttt{max\_trust\_region\_radius } ($10^{16}$)} The trust
  region radius is not allowed to grow beyond this value.

\item{\texttt{min\_trust\_region\_radius } ($10^{-32}$)} The solver
  terminates, when the trust region becomes smaller than this value.

\item{\texttt{min\_relative\_decrease }}($10^{-3}$) Lower threshold
  for relative decrease before a Levenberg-Marquardt step is acceped.

\item{\texttt{lm\_min\_diagonal } ($10^6$)} The
  \texttt{LEVENBERG\_MARQUARDT} strategy, uses a diagonal matrix to
  regularize the the trust region step. This is the lower bound on the
  values of this diagonal matrix.

\item{\texttt{lm\_max\_diagonal } ($10^{32}$)}  The
  \texttt{LEVENBERG\_MARQUARDT} strategy, uses a diagonal matrix to
  regularize the the trust region step. This is the upper bound on the
  values of this diagonal matrix.

\item{\texttt{max\_num\_consecutive\_invalid\_steps } (5)} The step
  returned by a trust region strategy can sometimes be numerically
  invalid, usually because of conditioning issues. Instead of crashing
  or stopping the optimization, the optimizer can go ahead and try
  solving with a smaller trust region/better conditioned problem. This
  parameter sets the number of consecutive retries before the
  minimizer gives up.

\item{\texttt{function\_tolerance }}($10^{-6}$) Solver terminates if
\begin{align}
\frac{|\Delta \text{cost}|}{\text{cost}} < \texttt{function\_tolerance}
\end{align}
where, $\Delta \text{cost}$ is the change in objective function value
(up or down) in the current iteration of Levenberg-Marquardt.

\item \texttt{Solver::Options::gradient\_tolerance } Solver terminates if
\begin{equation}
    \frac{\|g(x)\|_\infty}{\|g(x_0)\|_\infty} < \texttt{gradient\_tolerance}
\end{equation}
where $\|\cdot\|_\infty$ refers to the max norm, and $x_0$ is the vector of initial parameter values.

\item{\texttt{parameter\_tolerance }}($10^{-8}$) Solver terminates if
\begin{equation}
    \frac{\|\Delta x\|}{\|x\| + \texttt{parameter\_tolerance}} < \texttt{parameter\_tolerance}
\end{equation}
where $\Delta x$ is the step computed by the linear solver in the current iteration of Levenberg-Marquardt.

\item{\texttt{linear\_solver\_type }(\texttt{SPARSE\_NORMAL\_CHOLESKY})}

\item{\texttt{linear\_solver\_type
    }}(\texttt{SPARSE\_NORMAL\_CHOLESKY}/\texttt{DENSE\_QR}) Type of
  linear solver used to compute the solution to the linear least
  squares problem in each iteration of the Levenberg-Marquardt
  algorithm. If Ceres is build with \suitesparse linked in  then the
  default is \texttt{SPARSE\_NORMAL\_CHOLESKY}, it is
  \texttt{DENSE\_QR} otherwise.

\item{\texttt{preconditioner\_type }}(\texttt{JACOBI}) The
  preconditioner used by the iterative linear solver. The default is
  the block Jacobi preconditioner. Valid values are (in increasing
  order of complexity) \texttt{IDENTITY},\texttt{JACOBI},
  \texttt{SCHUR\_JACOBI}, \texttt{CLUSTER\_JACOBI} and
  \texttt{CLUSTER\_TRIDIAGONAL}.

\item{\texttt{sparse\_linear\_algebra\_library }
    (\texttt{SUITE\_SPARSE})} Ceres supports the use of two sparse
  linear algebra libraries, \texttt{SuiteSparse}, which is enabled by
  setting this parameter to \texttt{SUITE\_SPARSE} and
  \texttt{CXSparse}, which can be selected by setting this parameter
  to $\texttt{CX\_SPARSE}$. \texttt{SuiteSparse} is a sophisticated
  and complex sparse linear algebra library and should be used in
  general. If your needs/platforms prevent you from using
  \texttt{SuiteSparse}, consider using \texttt{CXSparse}, which is a
  much smaller, easier to build library. As can be expected, its
  performance on large problems is not comparable to that of
  \texttt{SuiteSparse}.


\item{\texttt{num\_linear\_solver\_threads }}(\texttt{1}) Number of
  threads used by the linear solver.

\item{\texttt{use\_inner\_iterations} (\texttt{false}) } Use a
  non-linear version of a simplified variable projection
  algorithm. Essentially this amounts to doing a further optimization
  on each Newton/Trust region step using a coordinate descent
  algorithm.  For more details, see the discussion in ~\ref{sec:inner}

\item{\texttt{inner\_iteration\_ordering} (\texttt{NULL})} If
  \texttt{Solver::Options::inner\_iterations} is true, then the user
  has two choices.

\begin{enumerate}
\item Let the solver heuristically decide which parameter blocks to
  optimize in each inner iteration. To do this set, this
  \texttt{inner\_iteration\_ordering} to
  {\texttt{NULL}}.

\item Specify a collection of of ordered independent sets. The lower
  numbered groups are optimized before the higher number groups during
  the inner optimization phase. Each group must be an independent set.
\end{enumerate}

\item{\texttt{ordering} (\texttt{NULL})} An instance of the ordering
  object informs the solver about the desired order in which parameter
  blocks should be eliminated by the linear solvers. See
  section~\ref{sec:ordering} for more details.

  If \texttt{NULL}, the solver is free to choose an ordering that it
  thinks is best. Note: currently, this option only has an effect on
  the Schur type solvers, support for the
  \texttt{SPARSE\_NORMAL\_CHOLESKY} solver is forth coming.

\item{\texttt{use\_block\_amd } (\texttt{true})} By virtue of the
  modeling layer in Ceres being block oriented, all the matrices used
  by Ceres are also block oriented.  When doing sparse direct
  factorization of these matrices, the fill-reducing ordering
  algorithms can either be run on the block or the scalar form of
  these matrices. Running it on the block form exposes more of the
  super-nodal structure of the matrix to the Cholesky factorization
  routines. This leads to substantial gains in factorization
  performance. Setting this parameter to true, enables the use of a
  block oriented Approximate Minimum Degree ordering
  algorithm. Settings it to \texttt{false}, uses a scalar AMD
  algorithm. This option only makes sense when using
  \texttt{sparse\_linear\_algebra\_library = SUITE\_SPARSE} as it uses
  the \texttt{AMD} package that is part of \texttt{SuiteSparse}.

\item{\texttt{linear\_solver\_min\_num\_iterations }}(\texttt{1})
  Minimum number of iterations used by the linear solver. This only
  makes sense when the linear solver is an iterative solver, e.g.,
  \texttt{ITERATIVE\_SCHUR}.

\item{\texttt{linear\_solver\_max\_num\_iterations }}(\texttt{500})
  Minimum number of iterations used by the linear solver. This only
  makes sense when the linear solver is an iterative solver, e.g.,
  \texttt{ITERATIVE\_SCHUR}.

\item{\texttt{eta }} ($10^{-1}$)
 Forcing sequence parameter. The truncated Newton solver uses this
 number to control the relative accuracy with which the Newton step is
 computed. This constant is passed to ConjugateGradientsSolver which
 uses it to terminate the iterations when
\begin{equation}
\frac{Q_i - Q_{i-1}}{Q_i} < \frac{\eta}{i}
\end{equation}

\item{\texttt{jacobi\_scaling }}(\texttt{true}) \texttt{true} means
  that the Jacobian is scaled by the norm of its columns before being
  passed to the linear solver. This improves the numerical
  conditioning of the normal equations.

\item{\texttt{logging\_type }}(\texttt{PER\_MINIMIZER\_ITERATION})


\item{\texttt{minimizer\_progress\_to\_stdout }}(\texttt{false})
By default the Minimizer progress is logged to \texttt{STDERR}
depending on the \texttt{vlog} level. If this flag is
set to true, and \texttt{logging\_type } is not \texttt{SILENT}, the
logging output
is sent to \texttt{STDOUT}.

\item{\texttt{return\_initial\_residuals }}(\texttt{false})
\item{\texttt{return\_final\_residuals }}(\texttt{false})
If true, the vectors \texttt{Solver::Summary::initial\_residuals } and
\texttt{Solver::Summary::final\_residuals } are filled with the
residuals before and after the optimization. The entries of these
vectors are in the order in which ResidualBlocks were added to the
Problem object.

\item{\texttt{return\_initial\_gradient }}(\texttt{false})
\item{\texttt{return\_final\_gradient }}(\texttt{false})
If true, the vectors \texttt{Solver::Summary::initial\_gradient } and
\texttt{Solver::Summary::final\_gradient } are filled with the
gradient before and after the optimization. The entries of these
vectors are in the order in which ParameterBlocks were added to the
Problem object.

Since \texttt{AddResidualBlock } adds ParameterBlocks to the
\texttt{Problem } automatically if they do not already exist, if you
wish to have explicit control over the ordering of the vectors, then
use \texttt{Problem::AddParameterBlock } to explicitly add the
ParameterBlocks in the order desired.

\item{\texttt{return\_initial\_jacobian }}(\texttt{false})
\item{\texttt{return\_initial\_jacobian }}(\texttt{false})
If true, the Jacobian matrices before and after the optimization are
returned in \texttt{Solver::Summary::initial\_jacobian } and
\texttt{Solver::Summary::final\_jacobian } respectively.

The rows of these matrices are in the same order in which the
ResidualBlocks were added to the Problem object. The columns are in
the same order in which the ParameterBlocks were added to the Problem
object.

Since \texttt{AddResidualBlock } adds ParameterBlocks to the
\texttt{Problem } automatically if they do not already exist, if you
wish to have explicit control over the column ordering of the matrix,
then use \texttt{Problem::AddParameterBlock } to explicitly add the
ParameterBlocks in the order desired.

The Jacobian matrices are stored as compressed row sparse
matrices. Please see \texttt{ceres/crs\_matrix.h } for more details of
the format.

\item{\texttt{lsqp\_iterations\_to\_dump }} List of iterations at
  which the optimizer should dump the linear least squares problem to
  disk. Useful for testing and benchmarking. If empty (default), no
  problems are dumped.

\item{\texttt{lsqp\_dump\_directory }} (\texttt{/tmp})
 If \texttt{lsqp\_iterations\_to\_dump} is non-empty, then this
 setting determines the directory to which the files containing the
 linear least squares problems are written to.


\item{\texttt{lsqp\_dump\_format }}(\texttt{TEXTFILE}) The format in
  which linear least squares problems should be logged
when \texttt{lsqp\_iterations\_to\_dump} is non-empty.  There are three options
\begin{itemize}
\item{\texttt{CONSOLE }} prints the linear least squares problem in a human readable format
  to \texttt{stderr}. The Jacobian is printed as a dense matrix. The vectors
   $D$, $x$ and $f$ are printed as dense vectors. This should only be used
   for small problems.
\item{\texttt{PROTOBUF }}
   Write out the linear least squares problem to the directory
   pointed to by \texttt{lsqp\_dump\_directory} as a protocol
   buffer. \texttt{linear\_least\_squares\_problems.h/cc} contains routines for
   loading these problems. For details on the on disk format used,
   see \texttt{matrix.proto}. The files are named
   \texttt{lm\_iteration\_???.lsqp}. This requires that
   \texttt{protobuf} be linked into Ceres Solver.
\item{\texttt{TEXTFILE }}
   Write out the linear least squares problem to the directory
   pointed to by \texttt{lsqp\_dump\_directory} as text files
   which can be read into \texttt{MATLAB/Octave}. The Jacobian is dumped as a
   text file containing $(i,j,s)$ triplets, the vectors $D$, $x$ and $f$ are
   dumped as text files containing a list of their values.

   A \texttt{MATLAB/Octave} script called \texttt{lm\_iteration\_???.m} is also output,
   which can be used to parse and load the problem into memory.
\end{itemize}



\item{\texttt{check\_gradients }}(\texttt{false})
 Check all Jacobians computed by each residual block with finite
     differences. This is expensive since it involves computing the
     derivative by normal means (e.g. user specified, autodiff,
     etc), then also computing it using finite differences. The
     results are compared, and if they differ substantially, details
     are printed to the log.

\item{\texttt{gradient\_check\_relative\_precision }} ($10^{-8}$)
  Relative precision to check for in the gradient checker. If the
  relative difference between an element in a Jacobian exceeds
  this number, then the Jacobian for that cost term is dumped.

\item{\texttt{numeric\_derivative\_relative\_step\_size }} ($10^{-6}$)
 Relative shift used for taking numeric derivatives. For finite
     differencing, each dimension is evaluated at slightly shifted
     values, \eg for forward differences, the numerical derivative is

\begin{align}
 \delta &= \texttt{numeric\_derivative\_relative\_step\_size}\\
 \Delta f &= \frac{f((1 + \delta)  x) - f(x)}{\delta x}
\end{align}

The finite differencing is done along each dimension. The
reason to use a relative (rather than absolute) step size is
that this way, numeric differentiation works for functions where
the arguments are typically large (e.g. $10^9$) and when the
values are small (e.g. $10^{-5}$). It is possible to construct
"torture cases" which break this finite difference heuristic,
but they do not come up often in practice.

\item{\texttt{callbacks }}
  Callbacks that are executed at the end of each iteration of the
\texttt{Minimizer}. They are executed in the order that they are
specified in this vector. By default, parameter blocks are
updated only at the end of the optimization, i.e when the
\texttt{Minimizer} terminates. This behavior is controlled by
\texttt{update\_state\_every\_variable}. If the user wishes to have access
to the update parameter blocks when his/her callbacks are
executed, then set \texttt{update\_state\_every\_iteration} to true.

The solver does NOT take ownership of these pointers.

\item{\texttt{update\_state\_every\_iteration }}(\texttt{false})
Normally the parameter blocks are only updated when the solver
terminates. Setting this to true update them in every iteration. This
setting is useful when building an interactive application using Ceres
and using an \texttt{IterationCallback}.

\item{\texttt{solver\_log}}  If non-empty, a summary of the execution of the solver is
 recorded to this file.  This file is used for recording and Ceres'
 performance. Currently, only the iteration number, total
 time and the objective function value are logged. The format of this
 file is expected to change over time as the performance evaluation
 framework is fleshed out.
\end{enumerate}

\section{\texttt{Solver::Summary}}
TBD
